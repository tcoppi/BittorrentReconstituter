\documentclass[11pt]{article}
\usepackage{eco}

\usepackage[letterpaper, bottom=1.5in, top=1.25in]{geometry}
\usepackage{enumerate}
\usepackage{url}

\usepackage{fancyhdr}
\pagestyle{fancy}
\lhead{CS 489---Network Forensics} \chead{October 13, 2009} \rhead{Proposal}
\cfoot{\thepage}

\author{Charlie Moore \and Aaron Lovato \and Thomas Coppi} 
\title{Proposal to Analyze and Retrieve Files from a Capture of BitTorrent
  Protocol Traffic} 
\date{October 20, 2009}

%===-------------------------------------------------------------------------===%
\begin{document}
\maketitle

\section{Introduction}
We propose to develop an application to identify BitTorrent file transfers in
network traffic and extract the original file.  We assume that a full and
complete packet capture has been performed.  We will not deal with encrypted
variants of BitTorrent. The application will read a packet capture file and
identify any BitTorrent file transfers contained in the file. We plan to design
this with the unix philosophy in mind, so it can also be used as a filter for
live traffic, \emph{\`a la} \verb=tcpdump=. If all packets for each file
transfer are found in the capture file, the application will rebuild the
original file according to the BitTorrent protocol specification.

The application will be originally developed for command-line usage; if time
permits, a GUI will be developed. Research has not yielded any applications
currently available that perform this task, so this would likely be the first
publicly available tool of its kind. One technique has been found that matches
BitTorrent transfers against known bad file
hashes\footnote{\url{http://www.technologyreview.com/computing/22107/?a=f}} but
it does not appear to actually retrieve the file.

\section{Threats (Use Cases)}
The main motivation for this project is to give an analyst investigating an
incident involving BitTorrent a tool to identify the role BitTorrent played.  An
example of such a situation would be an investigation searching for child
pornography transferred across a network. Without a tool of this sort, analysts
would have to manually reconstruct the files and do so in such a way as to not
compromise their admissibility in court. Furthermore, the analyst would have to
ensure that all parts were received by the computer in question. We will attempt
to verify the successful download of all parts of the file in order to verify
the integrity of the data our tool produces. 

Another situation in which this tool will be useful occurs when malware is
downloaded to a compromised computer using BitTorrent. If the attackers clean up
the the compromised computer before a disk image is taken, network traffic
provides the only source of a sample of the malware used. Retrieving this file
or files from the network would be useful for further mitigation of the
attack. Also, identifying the peers involved in the download would allow action
to be taken to prevent them from being used in further attacks. To this end, our
tool will also list all IP addresses involved in the transfer of each file.

One last obvious use case is to mitigate exfiltration of data.  If the user is
using the BitTorrent protocol to send data out of the network, we would like to
know.  This can be accomplished by inspecting the files that we extract, and
possibly ensuring that the headers don't contain sensitive information, either.
This will require using a live capture stream.

\section{Plan of Attack}
Our approach will start with using C++ along with \verb=libpcap= to read and
parse a packet capture file. Once all packets have been parsed into usable and
searchable data structures, our application will identify tracker
requests. Parsing tracker responses will be useful for identifying the peers
involved in the download, which can then be searched for in the captured
packets. The application will then attempt to decode these packets according to
the BitTorrent protocol specification. All ``piece'' messages will be stored in
a by-transfer fashion for later reconstruction of the transferred file. This
process will be repeated for every identified BitTorrent tracker request in the
capture file. Once all packets have been processed, the application will proceed
to rebuild the transferred files and output them to disk. Any file that is not
entirely contained in the capture file will be noted.  We will use a best-effort
reconstruction algorithm for partial captures.  The verification of the files
will depend on having the original torrent file.

\section{Risks}
The risk is that we might fail to extract the files from a capture file.  In
this event, we'll write up an analysis of why we can't do it, and what steps
need to be taken in order to complete the job.  We can fall back to simply
identifying traffic, but this has already been done.

\section{Deliverable}
Our application will analyze captured packets and extract the original file(s).
We will use the hashes provided in the protocol to verify the integrity of each
file.  We'll use the TCP session data to verify that the files were recieved.
We'll provide a list of all peers involved in the download.  We'll identify
which piece were uploaded by the target.

\end{document}
