\documentclass[11pt,titlepage]{article}
% Give us palatino and optima
%\usepackage[osf,sc]{mathpazo}
%\usepackage{palatino}
%\usepackage[T1]{fontenc}
%\renewcommand{\sfdefault}{uop}
\usepackage{eco}
\usepackage{url}
\usepackage{listings}
\usepackage{microtype}
\usepackage[letterpaper, bottom=1.5in, top=1.5in]{geometry}

\usepackage{fancyhdr}
\pagestyle{fancy}
\lhead{CS 489} \chead{\today} \rhead{Design Document}
\cfoot{\thepage}

\author{Charlie Moore \and Aaron Lovato \and Thomas Coppi}
\title{CS 489 --- BitTorrent Reconstituter Design Document}

\begin{document}
\maketitle

\section{Introduction}
This document is still fairly high level for now.  We're fleshing it out as we
go.  Our project will consist of two modules: the Pcap Parsing Module and the
File Reconstitution Module.  The former module must be run first and will parse
out the BitTorrent streams and pieces and do most of the grunt work.  The latter
module will do the actual work of verifying that pieces were received and
reconstructing the files that were found in the capture file.

\section{Pcap Parsing Module}
This module will parse the \texttt{pcap} file and give us BitTorrent
sessions.  It handles the protocol BitTorrent is running on, so all other
modules are protocol agnostic with respect to the underlying protocol. This
module will require two concurrent processes to be able to handle live input.
Process 1 will read in packets from \texttt{libpcap}, decode the TCP/IP headers,
discard non-TCP packets, and build a \texttt{Packet} data structure to hold the
relevant parts of the packet. These parts of the packet are the source and
destination ports, the IP addresses, and the TCP payload. This process will then
serialize the data and send it over a pipe to process 2.

Process 2 will read data from the pipe and re-create the corresponding
\texttt{Packet} data structures. It will then attempt to decode a tracker
request. If this packet does contain a request, it will be recorded. If not, the
process will then attempt to decode a tracker response. If a response is found,
the process will make sure a corresponding request was already found. If not,
the response will be discarded. Otherwise, relevant data will be pulled out of
the response (peers, ports, etc.) and stored.

If the packet is determined to contain neither a request nor a response, the
process will attempt to decode the payload as a BitTorrent protocol packet. 
Depending on the type of packet, the data may be stored or discarded. We still 
need to determine how exactly the data from BitTorrent packets will be stored.
Non-BitTorrent traffic will be discarded.

\subsection{Data Structure}
The Packet data structure has been completed. See Packet.hpp for details. One
uncompleted data structure is a vector of structs containing the session info.
\begin{lstlisting}[language=C++]
struct Session:
  vector<Peer> peers // peers which participated in this session
  vector<PieceSHA1> checksums // get from .torrent for verification
  vector<string> bt_msgs // list of protocol messages parsed and ready
\end{lstlisting}

\section{File Reconstruction Module}
This module will perform the work of rebuilding and verifying files based on a 
\texttt{.torrent} file and the data extracted from the network by the pcap 
parsing module. The following information is from our original design document, 
it may be useful for implementing the reconstruction module.

The next module will go through the sessions and build up a map for each file with
the pieces and other metadata that we need.

\subsection{Data Structure}
% XXX Need some discussion of this data structure here
\begin{lstlisting}[language=C++]
class File:
  vector<Map<PieceSHA1, PieceData> > pieces;
  // etc.
\end{lstlisting}

\end{document}
