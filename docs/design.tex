\documentclass[11pt,titlepage]{article}

% Give us palatino and optima
\usepackage[osf,sc]{mathpazo}
\usepackage{palatino}
\usepackage[T1]{fontenc}
\renewcommand{\sfdefault}{uop}

\usepackage{listings}
\usepackage{url}
\usepackage[letterpaper, bottom=1.5in, top=1.5in]{geometry}
\usepackage{fancyhdr}
\usepackage{microtype}

\pagestyle{fancy}
\lhead{CS 489} \chead{\today} \rhead{Design Document}
\cfoot{\thepage}

\author{Charlie Moore \and Aaron Lovato \and Thomas Coppi}
\title{CS 489 --- BitTorrent Reconstituter Design Document}

\begin{document}
\maketitle

\section{Introduction}
This document is still fairly high level for now.  We're fleshing it out as we
go.  Our project will consist of three modules.
% XXX Insert further discussion here.

\section{Pcap Parsing Module}
This first module will parse the \verb=pcap= file and give us BitTorrent
sessions.  It handles the protocol BitTorrent is running on, all other modules
are protocol agnostic with respect to the underlying protocol.

\subsection{Data Structure}
% XXX Need some discussion of this data structure here
Data structure is a vector of structs containing the session info.
\begin{lstlisting}[language=C++]
struct Session:
  vector<Peer> peers // peers which participated in this session
  vector<PieceSHA1> checksums // must get from .torrent for later verification
  vector<string> bt_msgs // list of protocol messages all parsed and ready
\end{lstlisting}

\section{File Finding Module}
The next module will go through the sessions and build up a map for each file with
the pieces and other metadata that we need.

\subsection{Data Structure}
% XXX Need some discussion of this data structure here
\begin{lstlisting}[language=C++]
class File:
  vector<Map<PieceSHA1, PieceData> > pieces;
  // other crap
\end{lstlisting}

\section{File Reconstituting Module}
The last module will actually reconstitute the files (and verify, etc)

\subsection{Data Structure}
Data structure is the finished file

\end{document}
